\documentclass[12pt]{article}

% --- Packages ---
\usepackage{amsmath, amssymb, amsthm}
\usepackage{graphicx}
\usepackage{booktabs}
\usepackage{hyperref}
\usepackage{geometry}
\usepackage{enumitem}
\usepackage{fancyhdr}
\usepackage{titlesec}
\usepackage{lmodern}
\usepackage{setspace}
\usepackage{abstract}
\usepackage{framed}

% --- Geometry ---
\geometry{a4paper, margin=1in}

% --- Header & Footer ---
\pagestyle{fancy}
\fancyhf{}
\lhead{\textbf{Your Name}}
\rhead{\textbf{Topic Title}}
\cfoot{\thepage}

% --- Title formatting ---
\titleformat{\section}{\large\bfseries}{\thesection.}{1em}{}
\titleformat{\subsection}{\normalsize\bfseries}{\thesubsection.}{1em}{}

% --- Spacing and Lists ---
\setstretch{1.2}
\setlist[itemize]{leftmargin=*, topsep=4pt}

% --- Document Title ---
\title{\textbf{\LARGE Title of the Topic Goes Here}}
\author{\Large Your Name}
\date{\today} % Or set a fixed date like: \date{June 2025}

\begin{document}

\maketitle

\begin{abstract}
\noindent
Write a concise summary of the note or derivation here. State the goal of the document and what concepts or results will be established. Keep it within 3–5 sentences.
\end{abstract}

\section{Introduction}

Provide an overview of the topic. Introduce key quantities, concepts, or notation that will be used. Keep it intuitive and motivating.

\begin{center}
\begin{framed}
\noindent\textbf{Review of [Insert Topic Here]} 

\vspace{0.5em}
Use this box to summarize a prior concept (e.g., OLS, probability axioms, etc.). These can include properties, formulas, or key steps. You can also include bullet points:

\begin{itemize}
    \item Key Point 1
    \item Key Point 2
    \item Key Formula: \( a^2 + b^2 = c^2 \)
\end{itemize}
\end{framed}
\end{center}

\section{Main Results}

\subsection{Algebraic Derivation}

Include derivation steps, algebraic identities, and proofs. Use equations:

\[
\text{Some Identity} = \text{Part 1} + \text{Part 2}
\]

\subsection{Geometric Interpretation}

Explain geometric intuition if applicable, like projection, orthogonality, or vector space relationships.

\[
\| \mathbf{a} \|^2 = \| \mathbf{b} \|^2 + \| \mathbf{c} \|^2
\]

\section{Conclusion}

Wrap up the key results, implications, or generalizations. You can also add:

\begin{center}
\begin{framed}
\centering
\boxed{
\text{Final Boxed Formula or Insight Goes Here}
}
\end{framed}
\end{center}

\end{document}
